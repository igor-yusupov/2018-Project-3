\documentclass[12pt,twoside]{article}
    \usepackage{jmlda}
    \usepackage{booktabs}
    \usepackage{csquotes}
    \usepackage{cite}
    \usepackage[backend=biber]{biblatex}
    \addbibresource{references.bib}
    \usepackage{multicol}
    
    
    %\NOREVIEWERNOTES
    \title
        {Кластеризация временных рядов}
    \author
        {Гончаров~А.\,В.,Юсупов~И.\,Р} % основной список авторов, выводимый в оглавление
    \thanks{
       % Работа выполнена при финансовой поддержке РФФИ, проект \No\,00-00-00000.
        %Научный руководитель:  Гончаров~А.\,В.
        Задачу поставил:  Гончаров~А.\,В.
        Консультант:  Гончаров~А.\,В.
    }
    \email
        {yusupov.ir@phystech.edu, alex.goncharov@phystech.edu}
    \organization
        {МФТИ}

    \abstract{
        В данной работе исследуется кластеризация временных рядов с использованием алгоритма DTW. При использовании DTW кластеризация временных рядов имеет большую временную сложность. Основной целью статьи является снижение временной сложности и  кластеризация временных рядов с помощью оптимизированной функции расстояния DTW.  

        \bigskip
        \textbf{Ключевые слова}: \emph {временные ряды, многомерные временные ряды, DTW}.
    }
    
    % \titleEng
    %     {JMLDA paper example: file jmlda-example.tex}
    % \authorEng
    %     {Author~F.\,S.$^1$, CoAuthor~F.\,S.$^2$, Name~F.\,S.$^2$}
    % \organizationEng
    %     {$^1$Organization; $^2$Organization}
    % \abstractEng
    %     {This document is an example of paper prepared with \LaTeXe\
    %     typesetting system and style file \texttt{jmlda.sty}.
    
    %     \bigskip
    %     \textbf{Keywords}: \emph{keyword, keyword, more keywords}.}
        
    \begin{document}
    \maketitle
    \section{Введение}
        Одной из актуальных задач анализа данных является задача кластеризации многомерных временных рядов. Такого рода задачи возникают при построении моделей объектов в трудноформализуемых областях
        исследований, например в медицине, когда требуется дать описание типичных групп пациентов со сходной динамикой развития заболевания на основе данных об изменениях клинических показателей и диагностических признаков. Типизация пациентов позволяет, в частности, разрабатывать методики лечения, оптимальные для каждой группы.\\
        В задаче кластерного анализа требуется разбить множество объектов, описываемых набором некоторых переменных или матрицей попарных расстояний на кластеры так, чтобы критерий качества принял наилучшее значение. Критерий качества – функционал, зависящий от разброса объектов  внутри кластера  и расстояний между кластерами. Определение расстояния или меры различия между временными рядами имеет дополнительные трудности: ряды могут быть разной длины, состоять из разнотипных компонентов, иметь большую размерность. Кроме того, предполагается наличие зависимостей между наблюдаемыми характеристиками в различные моменты времени.\\
        Евклидово расстояние имеет чувствительность к искажению по временной оси, поэтому для измерения расстояния между временными рядами используется функция расстояния DTW. \cite{berndt1994using}.  Гибкость этого метода позволяет оценить сходство двух временных рядов, несмотря на фазовые сдвиги.  DTW имеет вычислительную сложность $O(n^2)$, но тем не менее является лучшим известным решением для определения сходства между временными рядами. На сегодня предложено большое количество подходов для понижения вычислительной сложности: отбрасывание заведомо непохожих подпоследовательностей на основе оценки нижней границы расстояния \cite{ding2008querying, camerra2010isax},  индексирование \cite{lim2006using}, раннее прекращение заведомо нерезультативных вычислений \cite{rakthanmanon2012searching}.  В данной работе используется отбрасывание подпоследовательностей на основе оценки нижней границы расстояния (lower-bounding, LB). Это ускорение позволяет добиться вычислительной сложности $O(n)$\cite{rakthanmanon2012searching}.  Затем сравним алгоритмы кластеризации и выявим самый подходящий для кластеризации временных рядов с помощью DTW. Известно, что не все алгоритмы кластеризации подходят. Например, k-means не подходит, так как этот алгоритм настаивает на кластеризации всех элементов, в то время как это мешает точности ввиду того, что временные ряды не являются статическими и неоторые элементы из набора данных вовсе не должны быть собраны в кластер \cite{begum2015accelerating}, но есть оптимизация этого метода для DTW, которая называется RSTMF \cite{meesrikamolkul2012shape}. Также DBSCAN не подходит для DTW, так как DTW не является метрикой, то возникает сложность в индексации, особенно для многомерных рядов \cite{begum2015accelerating}. 
        \section{Постановка задачи}
        Имеется выборка $\mathbb{D} = \{(s_i, y_i)\}_i^m$, где $s_i \in \mathbb{S}$ – множество временных рядов, а  $y_i \in \mathbb{Y}$ – множество идентификаторов кластера. \\
        $\textbf{Опредедение 1}$. Алгоритм кластеризации — функция $a$: $X$ \arrow  $ Y$ \\
        Временные ряды не являются статическими данными, поэтому обычные методы кластеризации для них не всегда работают. Кластеризацию временных рядов можно осуществить двумя способами:\\
        1)  Для необработанных данных подобрать функцию расстояния так, чтобы известные методы кластеризации работали. \\
        2) Преобразовать данные в статические и использовать известные методы кластеризации. \\
        \\
        В данной работе мы используем первый способ и для этого вводим функцию расстояния DTW. 
        Качество метода кластеризации должна оцениваться некоторыми критериями. Выделяют две категории критериев в зависимости от того, известно ли количество кластеров или нет. \\
        Пусть количество кластеров известно и равно $k$ и пусть $G$ и $C$ – множества индентификаторов кластера, известные изначально и определенные алгоритмом кластеризации соответственно. Тогда вводят следующий критерий \\
        $\textbf{Опредедение 2}$. Мера сходства: \\
        \begin{center}
            $Sim(G,C)$ = $\frac{1}{k}\sum\limits_{i=1}^k\max_{1 \leq j \leq k} Sim(G_i, C_j)$\\ где $Sim(G_i, C_j) =\frac{2|G_i\cap C_j|}{|G_i| + |C_j|} $
        \end{center} \\
        Рассмотрим теперь случай,  когда количество кластеров неизвестно. Вводится множество $P_k$, которое обозначает множество всех кластеров, разбивающих множество временных рядов на $k$ кластеров. Критерий определяющий лучшую среди возможных группировок: \\
        \begin{center}
            $P(C^*) = \min_{C_j \in C \in P_k}\sum\limits_{j=1}^k p(C_j)$ \\
            где $p(C) = \frac{1}{2w(C)}\sum\limits_{X,Y\in C} w(X)w(Y)D(X,Y)$\\
            $w(X)$ - вес элемента $X$, $w(C) = \sum\limits_{X\in C} w(X)$ - вес всех элементов. \\
            $D(X,Y)$ - функция расстояния между элементами.
        \end{center} \\
    \section{Описание алгоритма}
    %    Пусть $A$ = ($a_1, ..., a_T$) и $B$  = ($b_1, ..., b_T$) 2 последовательности, а $\delta$ – расстояние между двумя элементами последовательности. Стоимость опитмального выравнивания вычисляется рекурсивно следующим образом: \\
    %   \begin{center}
    %        $D(A_i, B_j)$ = $\delta (a_i, b_j)$ + min($D(A_{i-1}, B_{j-1}), $ $D(A_i, B_{j-1})$, $D(A_{i-1}, B_j)$) , где $A_i$ – последовательность ($a_1, ..., a_i$) 
    %   \end{center}
    %    К сожалению, прямая реализация этого рекурсивного алгоритма имеет экспоненциальную стоимость по времени. 
        Для построения функции выравнивания и проверки её качества используются модель DTW (и её отптимазации).
    \paragraph{Описание функции расстояния между объектами}
        В данной работе в качестве метрического расстояния между объектами предлагается использовать строимость
        \textit{пути наименьшей стоимости} между объектами.
            
        Dynamic time warping - измерение расстояния между двумя временными рядами.
            
        Задано два временных ряда, $X$ длины $m_1$ и $Y$ длины $m_1$.
            \begin{align*}
                X &= x_1,x_2, ..., x_i, ..., x_{m_1} \\
                Y &= y_1,y_2, ..., y_j, ..., y_{m_2} \\
                & x_i, y_j \in \mathbb{R}^n
            \end{align*}
        Требуется построить матрицу размера $m_1\times m_2$ c элементами $D_{ij}=d(x_i, y_j)$, где d - выбранная метрика.
        Чтобы найти наибольшее соответсвие между рядами нужно найти выравнивающий путь W, который минимизирует расстояние между ними.
        W - набор смежных элементов матрицы D, $w_k = (i, j)_k$.
            $W = w_1,w_2, ..., w_k, ..., w_K $

            $max(m_1, m_2)\leq K \leq m_1 + m_2 + 1$, где K-длина выравнивающего пути
        Выравнивающий путь должен удовлетворять следующим условиям:
            \begin{enumerate}
                \item $w_1=(1,1)$, $w_K=(m_1, m_2)$
                \item $w_k = (a, b)$, $w_{k-1}=(a', b')$ : $a-a' \leq 1$, $b-b' \leq 1$ 
                \item $w_k = (a, b)$, $w_{k-1}=(a', b')$ : $a-a' \geq 0$, $b-b'\geq 0$
            \end{enumerate}

        Оптимальный выравнивающий путь должен минимизировать выравнивающую стоимость пути:
            $$
                DTW(X, Y)=\displaystyle\sum\limits_{k=1}^{K} w_k
            $$
            
        Путь находится рекуррентно:\\
            $\gamma(i, j) = d(q_i, c_j) + min({\gamma(i-1, j-1), \gamma(i-1, j), \gamma(i, j-1)})$ ,
            где $\gamma(i, j)$ суммарное расстояние, $d(q_i, c_j)$ расстояние в текущей клетке.
                        
                Кроме того, выравнивающий путь ограничивают тем, насколько он может отклоняться от диагонали.
                Типичным ограничением является полоса Сако-Чиба, в которой говорится, что путь искривления не может отклоняться от диагонали больше,
                чем на определённый процент клеток.
        
    \paragraph{Алгоритмы кластеризации}
        В данной работе используются следующие алгоритмы кластеризации: DBSCAN \cite{tran2013revised}, K-Means \cite{petitjean2011global} \cite{hartigan1979algorithm}, Hierarchical Clustering \cite{mullner2013fastcluster}, Agglomerative Clustering \cite{mullner2013fastcluster} \\
    \section{Базовый эксперимент}
        Цель эксперимента – оценить качество работы алгоритмов, используя функцию расстояния DTW, на небольшой выборке данных. \\
        В ходе эксперимента были использованы данные акселерометра. Они представляли собой временные ряды длинною в 600 точек. Из них была сгенерирована выборка из 90 рядов. Каждые 30 рядов принадлежат определенному классу. 
        
        \paragraph{Результаты}
        \begin{table}[h!]
        \begin{tabular}{l|l|l|l|c|}
        \cline{2-5}
                                           & \textbf{DBSCAN}           & \textbf{K-Means}          & \textbf{Hierarchical Clustering} & \textbf{Agglomerative Clustering} \\ \hline
        \multicolumn{1}{|l|}{\textbf{Sim}} & \multicolumn{1}{c|}{1.00} & \multicolumn{1}{c|}{1.00} & \multicolumn{1}{c|}{1.00}        & 0.47                              \\ \hline
        \end{tabular}
        \end{table}   
                
        \printbibliography

    
    % Решение Программного Комитета:
    %\ACCEPTNOTE
    %\AMENDNOTE
    %\REJECTNOTE
\end{document}